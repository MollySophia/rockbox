% $Id$ %
\chapter{Browsing and playing}
\section{\label{ref:file_browser}File Browser}
\screenshot{rockbox_interface/images/ss-file-browser}{The file browser}{}
Rockbox lets you browse your music in either of two ways. The 
\setting{File Browser} lets you navigate through the files and directories on 
your \dap, entering directories and executing the default action on each file.
To help differentiate files, each file format is displayed with an icon. 

The \setting{Database Browser}, on the other hand, allows you to navigate 
through the music on your player using categories like album, artist, genre,
etc.

You can select whether to browse using the \setting{File Browser} or the
\setting{Database Browser} by selecting either \setting{Files} or
\setting{Database} in the \setting{Main Menu}.
If you choose the \setting{File Browser}, the \setting{Show Files} setting
lets you select what types of files you wish to view. See
\reference{ref:ShowFiles} for more information on the \setting{Show Files}
setting.

\note{The \setting{File Browser} allows you to manipulate your files in ways
that are not available within the \setting{Database Browser}. Read more about
\setting{Database} in \reference{ref:database}. The remainder of this section
deals with the \setting{File Browser}.}

\opt{iriverh10,iriverh10_5gb}{\note{
If your \dap{} is a MTP model, the Music directory where all your music is stored
may be hidden in the \setting{File Browser}. This may be fixed by either
changing its properties (on a computer) to not hidden, or by changing
the \setting{Show Files} setting to all.
}}

\subsection{\label{ref:controls}File Browser Controls}
\begin{btnmap}
      \ActionStdPrev{}/\ActionStdNext{}
      \opt{HAVEREMOTEKEYMAP}{& \ActionRCStdPrev{}/\ActionRCStdNext{}}
         & Go to previous/next item in list. If you are on the first/last 
           entry, the cursor will wrap to the last/first entry.\\
      %
      \opt{IRIVER_H100_PAD,IRIVER_H300_PAD}
        {
          \ButtonOn+\ButtonUp{}/ \ButtonDown
          \opt{HAVEREMOTEKEYMAP}{&
            \opt{IRIVER_RC_H100_PAD}{\ButtonRCSource{}/ \ButtonRCBitrate}
          }
          & Move one page up/down in the list.\\
        }
      \opt{IRIVER_H10_PAD}
        {
          \ButtonRew{}/ \ButtonFF
          & Move one page up/down in the list.\\
        }
      %
      \ActionTreeParentDirectory
      \opt{HAVEREMOTEKEYMAP}{& \ActionRCTreeParentDirectory}
      & Go to the parent directory.\\
      %
      \ActionTreeEnter
      \opt{HAVEREMOTEKEYMAP}{& \ActionRCTreeEnter}
      & Execute the default action on the selected file or enter a
        directory.\\
      %
      \ActionTreeWps 
      \opt{HAVEREMOTEKEYMAP}{& \ActionRCTreeWps}
         & If there is an audio file playing, return to the
         \setting{While Playing Screen} (WPS) without stopping playback.\\
      %
      \nopt{player,SANSA_C200_PAD}%
        {%
          \ActionTreeStop 
          \opt{HAVEREMOTEKEYMAP}{& \ActionRCTreeStop}
          & Stop audio playback.\\%
        }%
      %
      \ActionStdContext{}
      \opt{HAVEREMOTEKEYMAP}{& \ActionRCStdContext}
      & Enter the \setting{Context Menu}.\\
      %
      \ActionStdMenu{}
      \opt{HAVEREMOTEKEYMAP}{& \ActionRCStdMenu}
      & Enter the \setting{Main Menu}.\\
      %
      \opt{quickscreen}{
        \ActionStdQuickScreen
        \opt{HAVEREMOTEKEYMAP}{& \ActionRCStdQuickScreen}
        & Switch to the \setting{Quick Screen}
        (see \reference{ref:QuickScreen}). \\
      }
      %
      \opt{SANSA_E200_PAD}{
        \ActionStdRec & Switch to the \setting{Recording Screen}.\\
      %
      }
      \nopt{touchscreen}{\opt{hotkey}{
        \ActionTreeHotkey
            &
        \opt{HAVEREMOTEKEYMAP}{
            &}
        Activate the \setting{Hotkey} function
        (see \reference{ref:Hotkeys}).
            \\
      }}
\end{btnmap}

\subsection{\label{ref:Contextmenu}\label{ref:PartIISectionFM}Context Menu}
\screenshot{rockbox_interface/images/ss-context-menu}{The Context Menu}{}

The \setting{Context Menu} allows you to perform certain operations on files or 
directories.  To access the \setting{Context Menu}, position the selector over a file 
or directory and access the context menu with \ActionStdContext{}.\\

\note{The \setting{Context Menu} is a context sensitive menu.  If the 
\setting{Context Menu} is invoked on a file, it will display options available 
for files.  If the \setting{Context Menu} is invoked on a directory, 
it will display options for directories.\\}

The \setting{Context Menu} contains the following options (unless otherwise noted, 
each option pertains both to files and directories):

\begin{description}
\item [View.]
  Displays the contents of the selected playlist file.
\item [Current Playlist.]
  Enters the \setting{Current Playlist Submenu} (see \reference{ref:currentplaylist_submenu}).
\item [Playlist Catalogue.]
  Enters the \setting{Playlist Catalogue Submenu} (see 
  \reference{ref:playlist_catalogue}).
\item [Rename.]
  This function lets the user modify the name of a file or directory.
\item [Cut.]
  Copies the name of the currently selected file or directory to the clipboard
  and marks it to be `cut'.
\item [Copy.]
  Copies the name of the currently selected file or directory to the clipboard
  and marks it to be `copied'.
\item [Paste.]
  Only visible if a file or directory name is on the clipboard. When selected
  it will move or copy the clipboard to the current directory.
\item [Delete.]
  Deletes the currently selected file. This option applies only to files, and
  not to directories. Rockbox will ask for confirmation before deleting a file.
  Press \ActionYesNoAccept{}
  to confirm deletion or any other key to cancel.
\item [Delete Directory.]
  Deletes the currently selected directory and all of the files and subdirectories
  it may contain. Deleted directories cannot be recovered. Use this feature with
  caution!
\opt{lcd_non-mono}{
\item [Set As Backdrop.]
  Set the selected \fname{bmp} file as background image. The bitmaps need to meet the
  conditions explained in \reference{ref:LoadingBackdrops}.
}
\item [Open with.]
  Runs a viewer plugin on the file. Normally, when a file is selected in Rockbox,
  Rockbox automatically detects the file type and runs the appropriate plugin.
  The \setting{Open With} function can be used to override the default action and
  select a viewer by hand.
  For example, this function can be used to view a text file
  even if the file has a non-standard extension (i.e., the file has an extension
  of something other than \fname{.txt}). See \reference{ref:Viewersplugins}
  for more details on viewers.
\item [Create Directory.]
  Create a new directory in the current directory on the disk.
\item [Properties.]
  Shows properties such as size and the time and date of the last modification
  for the selected file. If used on a directory, the number of files and
  subdirectories will be shown, as well as the total size.
\opt{recording}{
  \item [Set As Recording Directory.]
    Save recordings in the selected directory.
}
\item [\label{ref:StartFileBrowserHere}Start File Browser Here.]
  This option allows users to set the currently selected directory as the default
  start directory for the file browser. This option is not available for files.
  \note{If you have \setting{Auto-Change Directory} and
  \setting{Constrain Auto-Change} enabled, the directories returned will
  be constrained to the directory you have chosen here and those below it.
  See \reference{ref:ConstrainAutoChange}}
\item [Add to Shortcuts.]
  Adds a link to the selected item in the \fname{shortcuts.link} file.
  If the file does not already exist it will be created in the root directory.
  Note that if you create a shortcut to a file, Rockbox will not open it upon
  selecting, but simply bring you to its location in the \setting{File Browser}.
\end{description}

\subsection{\label{sec:virtual_keyboard}Virtual Keyboard}
\screenshot{rockbox_interface/images/ss-virtual-keyboard}{The virtual keyboard}{}
This is the virtual keyboard that is used when entering text in Rockbox, for 
example when renaming a file or creating a new directory.
The virtual keyboard can be easily changed by making a text file
with the required layout. More information on how to achieve this can be found
on the Rockbox website at \wikilink{LoadableKeyboardLayouts}.

\opt{morse_input}{
  Also you can switch to Morse code input mode by changing the
  \setting{Use Morse Code Input} setting%
  \opt{IRIVER_H100_PAD,IRIVER_H300_PAD,IPOD_4G_PAD,IPOD_3G_PAD,IRIVER_H10_PAD%
      ,GIGABEAT_PAD,GIGABEAT_S_PAD,MROBE100_PAD,SANSA_E200_PAD,PBELL_VIBE500_PAD%
      ,SANSA_FUZEPLUS_PAD,SAMSUNG_YH92X_PAD,SAMSUNG_YH820_PAD}
    { or by pressing \ActionKbdMorseInput{} in the virtual keyboard}%
  .}

% no "Actions" yet in the Player's virtual keyboard

\note{When the cursor is on the input line, \ActionKbdSelect{} deletes the preceding character}

\begin{btnmap}
    \opt{IRIVER_H100_PAD,IRIVER_H300_PAD,GIGABEAT_PAD,GIGABEAT_S_PAD%
        ,MROBE100_PAD,SANSA_E200_PAD,SANSA_FUZE_PAD,SANSA_C200_PAD,SANSA_FUZEPLUS_PAD%
        ,SAMSUNG_YH820_PAD}{
        \ActionKbdCursorLeft{} / \ActionKbdCursorRight
            &
        \opt{HAVEREMOTEKEYMAP}{\ActionRCKbdCursorLeft{} / \ActionRCKbdCursorRight
            &}
        Move the line cursor within the text line.
            \\
        %
        \ActionKbdBackSpace
            &
        \opt{HAVEREMOTEKEYMAP}{
            &}
        Delete the character before the line cursor.
            \\
    }%
    \ActionKbdLeft{} / \ActionKbdRight
        &
    \opt{HAVEREMOTEKEYMAP}{\ActionRCKbdLeft{} / \ActionRCKbdRight
        &}
    Move the cursor on the virtual keyboard.
    If you move out of the picker area, you get the previous/next page of
    characters (if there is more than one).
        \\
    %
    \ActionKbdUp{} / \ActionKbdDown
        &
    \opt{HAVEREMOTEKEYMAP}{\ActionRCKbdUp{} / \ActionRCKbdDown
        &}
    Move the cursor on the virtual keyboard.
    If you move out of the picker area you get to the line edit mode.
        \\
    %
    \nopt{IPOD_3G_PAD,IPOD_4G_PAD,IRIVER_H10_PAD,PBELL_VIBE500_PAD%
         ,SANSA_FUZEPLUS_PAD,SAMSUNG_YH92X_PAD,SAMSUNG_YH820_PAD}{
        \ActionKbdPageFlip
            &
        \opt{HAVEREMOTEKEYMAP}{\ActionRCKbdPageFlip
            &}
        Flip to the next page of characters (if there is more than one).
            \\
    }
    %
    \ActionKbdSelect
        &
    \opt{HAVEREMOTEKEYMAP}{\ActionRCKbdSelect
        &}
    Insert the selected keyboard letter at the current line cursor position.
        \\
    %
    \ActionKbdDone
        &
    \opt{HAVEREMOTEKEYMAP}{\ActionRCKbdDone
        &}
    Exit the virtual keyboard and save any changes.
        \\
    %
    \ActionKbdAbort
        &
    \opt{HAVEREMOTEKEYMAP}{\ActionRCKbdAbort
        &}
    Exit the virtual keyboard without saving any changes.
        \\
% to be done - create a separate section for morse imput and update the info
      \opt{morse_input}{
        \opt{IRIVER_H100_PAD,IRIVER_H300_PAD,GIGABEAT_PAD,GIGABEAT_S_PAD,MROBE100_PADD%
            ,SANSA_E200_PA,IPOD_4G_PAD,IPOD_3G_PAD,IRIVER_H10_PAD,PBELL_VIBE500_PAD%
            ,SAMSUNG_YH92X_PAD,SAMSUNG_YH820_PAD}{
          \ActionKbdMorseInput
          \opt{HAVEREMOTEKEYMAP}{& \ActionRCKbdMorseInput}
          & Toggle keyboard input mode and Morse code input mode. \\}
        %
        \ActionKbdMorseSelect
        \opt{HAVEREMOTEKEYMAP}{& \ActionRCKbdMorseSelect}
        & Tap to select a character in Morse code input mode. \\
      } 
\end{btnmap}

% $Id$ %
\section{\label{ref:database}Database}

\subsection{Introduction}
This chapter describes the Rockbox music database system. Using the information
contained in the tags (ID3v1, ID3v2, Vorbis Comments, Apev2, etc.) in your
audio files, Rockbox builds and maintains a database of the music
files on your player and allows you to browse them by Artist, Album, Genre, 
Song Name, etc.  The criteria the database uses to sort the songs can be completely
 customised. More information on how to achieve this can be found on the Rockbox
 website at \wikilink{DataBase}. 

\subsection{Initializing the Database}
The first time you use the database, Rockbox will scan your disk for audio files.
This can take quite a while depending on the number of files on your \dap{}.
This scan happens in the background, so you can choose to return to the
Main Menu and continue to listen to music.
If you shut down your player, the scan will continue next time you turn it on.
After the scan is finished you may be prompted to restart your \dap{} before
you can use the database.

\subsubsection{Ignoring Directories During Database Initialization}

You may have directories on your \dap{} whose contents should not be added
to the database. Placing a file named \fname{database.ignore} in a directory
will exclude the files in that directory and all its subdirectories from
scanning their tags and adding them to the database. This will speed up the
database initialization.

If a subdirectory of an `ignored' directory should still be scanned, place a
file named \fname{database.unignore} in it. The files in that directory and
its subdirectories will be scanned and added to the database.

\subsection{\label{ref:databasemenu}The Database Menu}

\begin{description}
  \opt{tc_ramcache}{
  \item[Load To RAM]
    The database can either be kept on \disk{} (to save memory), or
    loaded into RAM (for fast browsing). Setting this to \setting{Yes} loads
    the database to RAM, allowing faster browsing and searching. Setting this
    option to \setting{No} keeps the database on the \disk{}, meaning slower 
    browsing but it does not use extra RAM and saves some battery on boot up. 
    
    \opt{HAVE_DISK_STORAGE}{
    \note{If you browse your music frequently using the database, you should
      load to RAM, as this will reduce the overall battery consumption because
      the disk will not need to spin on each search.}
    }
  }
  
\item[Auto Update]
  If \setting{Auto update} is set to \setting{on}, each time the \dap{}
  boots, the database will automatically be updated.

\item[Initialize Now]
  You can force Rockbox to rescan your disk for tagged files by
  using the \setting{Initialize Now} function in the \setting{Database
    Menu}.
  \warn{\setting{Initialize Now} removes all database files (removing
    runtimedb data also) and rebuilds the database from scratch.}

\item[Update Now]
  \setting{Update now} causes the database to detect new and deleted files
    \note{Unlike the \setting{Auto Update} function, \setting{Update Now}
      will update the database regardless of whether the \setting{Directory Cache}
      is enabled. Thus, an update using \setting{Update now} may take a long
      time.
  }
  Unlike \setting{Initialize Now}, the \setting{Update Now} function
  does not remove runtime database information.
  
\item[Gather Runtime Data]
  When enabled, rockbox will record how often and how long a track is being played, 
  when it was last played and its rating. This information can be displayed in
  the WPS and is used in the database browser to, for example, show the most played, 
  unplayed and most recently played tracks.
  
\item[Export Modifications]
  This allows for the runtime data to be exported to the file \\
  \fname{/.rockbox/database\_changelog.txt}, which backs up the runtime data in
  ASCII format. This is needed when database structures change, because new
  code cannot read old database code. But, all modifications
  exported to ASCII format should be readable by all database versions.
  
\item[Import Modifications.]
  Allows the \fname{/.rockbox/database\_changelog.txt} backup to be 
  conveniently loaded into the database. If \setting{Auto Update} is
  enabled this is performed automatically when the database is initialized.
  
\end{description}

\subsection{Using the Database}
Once the database has been initialized, you can browse your music 
by Artist, Album, Genre, Song Name, etc.  To use the database, go to the
 \setting{Main Menu} and select \setting{Database}.\\

\note{You may need to increase the value of the \setting{Max Entries in File
Browser} setting (\setting{Settings $\rightarrow$ General Settings
$\rightarrow$ System $\rightarrow$ Limits}) in order to view long lists of
tracks in the ID3 database browser.\\

There is no option to turn off database completely. If you do not want
to use it just do not do the initial build of the database and do not load it
to RAM.}%

\begin{table}
  \begin{rbtabular}{.75\textwidth}{XXX}%
  {\textbf{Tag}   & \textbf{Type}  & \textbf{Origin}}{}{}
  filename              & string    & system \\ 
  album                 & string    & id tag \\
  albumartist           & string    & id tag \\
  artist                & string    & id tag \\
  comment               & string    & id tag \\
  composer              & string    & id tag \\
  genre                 & string    & id tag \\
  grouping              & string    & id tag \\
  title                 & string    & id tag \\
  bitrate               & numeric   & id tag \\
  discnum               & numeric   & id tag \\
  year                  & numeric   & id tag \\
  tracknum              & numeric   & id tag/filename \\
  autoscore             & numeric   & runtime db \\
  lastplayed            & numeric   & runtime db \\
  playcount             & numeric   & runtime db \\
  Pm (play time -- min)  & numeric   & runtime db \\
  Ps (play time -- sec)  & numeric   & runtime db \\
  rating                & numeric   & runtime db \\
  commitid              & numeric   & system \\
  entryage              & numeric   & system \\
  length                & numeric   & system \\
  Lm (track len -- min)  & numeric   & system \\
  Ls (track len -- sec)  & numeric   & system \\
  \end{rbtabular}
\end{table}

% $Id$ %
\section{\label{ref:WPS}While Playing Screen}
The While Playing Screen (WPS) displays various pieces of information about the
currently playing audio file.
%
The appearance of the WPS can be configured using WPS configuration files.
The items shown depend on your configuration -- all items can be turned on
or off independently. Refer to \reference{ref:wps_tags} for details on how
to change the display of the WPS.
\begin{itemize}
\item Status bar: The Status bar shows Battery level, charger status,
  volume, play mode, repeat mode, shuffle mode\opt{rtc}{ and clock}.
  In contrast to all other items, the status bar is always at the top of
  the screen.
\item (Scrolling) path and filename of the current song.
\item The ID3 track name.
\item The ID3 album name.
\item The ID3 artist name.
\item Bit rate. VBR files display average bitrate and ``(avg)''
\item Elapsed and total time.
\item A slidebar progress meter representing where in the song you are.
\item Peak meter.
\end{itemize}
%

See \reference{ref:ConfiguringtheWPS} for details of customising
your WPS (While Playing Screen).


\subsection{\label{ref:WPS_Key_Controls}WPS Key Controls}

  \begin{btnmap}
      \ActionWpsVolUp{} / \ActionWpsVolDown
      \opt{HAVEREMOTEKEYMAP}{& \ActionRCWpsVolUp{} / \ActionRCWpsVolDown}
      & Volume up/down.\\
      %
      \ActionWpsSkipPrev
       \opt{HAVEREMOTEKEYMAP}{& \ActionRCWpsSkipPrev}
      & Go to beginning of track, or if pressed while in the
        first seconds of a track, go to the previous track.\\
      %
      \ActionWpsSeekBack
      \opt{HAVEREMOTEKEYMAP}{& \ActionRCWpsSeekBack}
      & Rewind in track.\\
      %
      \ActionWpsSkipNext
      \opt{HAVEREMOTEKEYMAP}{& \ActionRCWpsSkipNext}
      & Go to the next track.\\
      %
      \ActionWpsSeekFwd
      \opt{HAVEREMOTEKEYMAP}{& \ActionRCWpsSeekFwd}
      & Fast forward in track.\\
      %
      \ActionWpsPlay
      \opt{HAVEREMOTEKEYMAP}{& \ActionRCWpsPlay}
      & Toggle play/pause.\\
      %
      \ActionWpsStop
      \opt{HAVEREMOTEKEYMAP}{& \ActionRCWpsStop}
      & Stop playback.\\
      %
      \ActionWpsBrowse
      \opt{HAVEREMOTEKEYMAP}{& \ActionRCWpsBrowse}
      & Return to the \setting{File Browser} / \setting{Database}.\\
      %
      \ActionWpsContext
      \opt{HAVEREMOTEKEYMAP}{& \ActionRCWpsContext}
      & Enter \setting{WPS Context Menu}.\\
      %
      \ActionWpsMenu
      \opt{HAVEREMOTEKEYMAP}{& \ActionRCWpsMenu}
      & Enter \setting{Main Menu}%
      .\\%
      %
      \opt{quickscreen}{%
        \ActionWpsQuickScreen
        \opt{HAVEREMOTEKEYMAP}{& \ActionRCWpsQuickScreen}
          & Switch to the \setting{Quick Screen}
          (see \reference{ref:QuickScreen}). \\}%
      %
      % software hold targets
      \nopt{hold_button}{%
          \opt{SANSA_CLIP_PAD}{\ButtonHome+\ButtonSelect}
          \opt{SANSA_FUZEPLUS_PAD}{\ButtonPower}
          & Key lock (software hold switch) on/off.\\
      }%
      % We explicitly list all the appropriate targets here and do no condition
      % on the 'pitchscreen' feature since some players have the feature but do
      % not have the button to go from the WPS to the pitch screen.
      \opt{IRIVER_H100_PAD,IRIVER_H300_PAD,IRIVER_H10_PAD,MROBE100_PAD%
          ,GIGABEAT_PAD,GIGABEAT_S_PAD,SANSA_E200_PAD,SANSA_C200_PAD,SANSA_FUZEPLUS_PAD}{%
        \ActionWpsPitchScreen
        \opt{HAVEREMOTEKEYMAP}{& \ActionRCWpsPitchScreen}
          & Show \setting{Pitch Screen} (see \reference{sec:pitchscreen}).\\%
      }%
      \opt{GIGABEAT_PAD,GIGABEAT_S_PAD,SANSA_CLIP_PAD,MROBE100_PAD,PBELL_VIBE500_PAD%
          ,SAMSUNG_YH92X_PAD,SAMSUNG_YH820_PAD,XDUOO_X3_PAD}{%
        \ActionWpsPlaylist
        \opt{HAVEREMOTEKEYMAP}{&}
          & Show current \setting{Playlist}.\\%
      }%
      \opt{IRIVER_H100_PAD,IRIVER_H300_PAD,IRIVER_H10_PAD%
          ,SANSA_E200_PAD,SANSA_C200_PAD,SANSA_FUZEPLUS_PAD}{%
        \ActionWpsIdThreeScreen
          \opt{HAVEREMOTEKEYMAP}{& \ActionRCWpsIdThreeScreen}
          & Enter \setting{ID3 Viewer}.\\%
      }%
      \opt{hotkey}{%
        \ActionWpsHotkey \opt{HAVEREMOTEKEYMAP}{& }
        & Activate the \setting{Hotkey} function (see \reference{ref:Hotkeys}).\\
      }
      \opt{ab_repeat_buttons}{%
         \ActionWpsAbSetBNextDir{} or }%
         % not all targets have the above action defined but the one below works on all
      Short \ActionWpsSkipNext{} + Long \ActionWpsSkipNext
      \opt{HAVEREMOTEKEYMAP}{
        &
          \opt{IRIVER_RC_H100_PAD}{\ActionRCWpsAbSetBNextDir{} or}
        Short \ActionRCWpsSkipNext{} + Long \ActionRCWpsSkipNext}
      & Skip to the next directory.\\
      %
      \opt{ab_repeat_buttons}{%
         \ActionWpsAbSetAPrevDir{} or }%
      Short \ActionWpsSkipPrev{} + Long \ActionWpsSkipPrev
      \opt{HAVEREMOTEKEYMAP}{
        &
          \opt{IRIVER_RC_H100_PAD}{\ActionRCWpsAbSetAPrevDir{} or}
        Short \ActionRCWpsSkipPrev{} + Long \ActionRCWpsSkipPrev}
      & Skip to the previous directory.\\
      %
      \opt{SANSA_E200_PAD,SANSA_C200_PAD,IRIVER_H100_PAD,IRIVER_H300_PAD}{
        \ActionStdRec
          \opt{HAVEREMOTEKEYMAP}{&}
          & Switch to the \setting{Recording Screen}.\\
      }%
  \end{btnmap}


\subsection{\label{ref:peak_meter}Peak Meter}
The peak meter can be displayed on the While Playing Screen and consists of
several indicators.
\opt{recording}{
  For a picture of the peak meter, please see the While
  Recording Screen in \reference{ref:while_recording_screen}.
}
\opt{ipodvideo}{
  \note{Especially the \playerman{} \playertype{}'s performance and battery runtime
   suffers when this feature is enabled. For this \dap{} it is highly recommended
   to not use peak meter.}
}

\begin{description}
\item [The bar:]
  This is the wide horizontal bar. It represents the current volume value.
\item [The peak indicator:]
  This is a little vertical line at the right end of the bar. It indicates
  the peak volume value that occurred recently.
\item [The clip indicator:]
  This is a little black block that is displayed at the very right of the
  scale when an overflow occurs. It usually does not show up during normal
  playback unless you play an audio file that is distorted heavily.
  \opt{recording}{
    If you encounter clipping while recording, your recording will sound distorted.
    You should lower the gain.
  }
  \note{Note that the clip detection is not very precise.
   Clipping might occur without being indicated.}
\item [The scale:]
  Between the indicators of the right and left channel there are little dots.
  These dots represent important volume values. In linear mode each dot is a
  10\% mark. In dBFS mode the dots represent the following values (from right
  to left): 0~dB, {}-3~dB, {}-6~dB, {}-9~dB, {}-12~dB, {}-18~dB, {}-24~dB, {}-30~dB,
  {}-40~dB, {}-50~dB, {}-60~dB.
\end{description}

\subsection{\label{sec:contextmenu}The WPS Context Menu}
Like the context menu for the \setting{File Browser}, the \setting{WPS Context Menu}
allows you quick access to some often used functions.

\subsubsection{Playlist}
The \setting{Playlist} submenu allows you to view, save, search, reshuffle,
and display the play time of the current playlist. These and other operations
are detailed in \reference{ref:working_with_playlists}. To change settings for
the \setting{Playlist Viewer} press \ActionStdContext{} while viewing the
current playlist to bring up the \setting{Playlist Viewer Menu}. In this
menu, you can find the \setting{Playlist Viewer Settings}.

\paragraph{Playlist Viewer Settings}
  \begin{description}
    \item[Show Icons.] This toggles display of the icon for the currently
    selected playlist entry and the icon for moving a playlist entry
    \item[Show Indices.] This toggles display of the line numbering for
       the playlist
    \item[Track Display.] This toggles between filename only and full path
       for playlist entries
  \end{description}


\subsubsection{Playlist catalogue}
  \begin{description}
    \item [Add to playlist.] Adds the currently playing file to a playlist.
    Select the playlist you want the file to be added to and it will get
    appended to that playlist.
    \item [Add to new playlist.] Similar to the previous entry this will
    add the currently playing track to a playlist. You need to enter a name
    for the new playlist first.
  \end{description}

\subsubsection{Sound Settings}
This is a shortcut to the \setting{Sound Settings Menu}, where you can configure volume,
bass, treble, and other settings affecting the sound of your music.
See \reference{ref:configure_rockbox_sound} for more information.

\subsubsection{Playback Settings}
This is a shortcut to the \setting{Playback Settings Menu}, where you can configure shuffle,
repeat, party mode, skip length and other settings affecting the playback of your music.

\subsubsection{Rating}
The menu entry is only shown if \setting{Gather Runtime Information} is
enabled. It allows the assignment of a personal rating value (0 -- 10)
to a track which can be displayed in the WPS and used in the Database
browser. The value wraps at 10.

\subsubsection{Bookmarks}
This allows you to create a bookmark in the currently-playing track.

\subsubsection{\label{ref:trackinfoviewer}Show Track Info}
\screenshot{rockbox_interface/images/ss-id3-viewer}{The track info viewer}{}
This screen is accessible from the WPS screen, and provides a detailed view of
all the identity information about the current track. This info is known as
meta data and is stored in audio file formats to keep information on artist,
album etc. To access this screen, %
\opt{IRIVER_H100_PAD,IRIVER_H300_PAD,IRIVER_H10_PAD,%
      SANSA_C200_PAD,SANSA_E200_PAD,SANSA_FUZE_PAD,SANSA_FUZEPLUS_PAD}{
  press \ActionWpsIdThreeScreen. }%
\opt{IPOD_4G_PAD,IPOD_3G_PAD,IAUDIO_X5_PAD,IAUDIO_M3_PAD,%
      GIGABEAT_PAD,GIGABEAT_S_PAD,MROBE100_PAD,SANSA_CLIP_PAD,PBELL_VIBE500_PAD,%
      MPIO_HD200_PAD,MPIO_HD300_PAD,SAMSUNG_YH92X_PAD,SAMSUNG_YH820_PAD,XDUOO_X3_PAD}%
      {press \ActionWpsContext{} to access the
      \setting{WPS Context Menu} and select \setting{Show Track Info}. }

\subsubsection{Open With...}
This \setting{Open With} function is the same as the \setting{Open With}
function in the file browser's \setting{Context Menu}.

\subsubsection{Delete}
Delete the currently playing file. The file will be deleted but the playback
of the file will not stop immediately. Instead, the part of the file that
has already been buffered (i.e. read into the \daps\ memory) will be played.
This may even be the whole track.

\opt{pitchscreen}{
  \subsubsection{\label{sec:pitchscreen}Pitch}

  The \setting{Pitch Screen} allows you to change the rate of playback
  (i.e. the playback speed and at the same time the pitch) of your
  \dap.  The rate value can be adjusted
  between 50\% and 200\%. 50\% means half the normal playback speed and a
  pitch that is an octave lower than the normal pitch. 200\% means double
  playback speed and a pitch that is an octave higher than the normal pitch.

  The rate can be changed in two modes: procentual and semitone.
  Initially, procentual mode is active.

    If you've enabled the \setting{Timestretch} option in
    \setting{Sound Settings} and have since rebooted, you can also use
    timestretch mode. This allows you to change the playback speed
    without affecting the pitch, and vice versa.

    In timestretch mode there are separate displays for pitch and
    speed, and each can be altered independently.  Due to the
    limitations of the algorithm, speed is limited to be between 35\%
    and 250\% of the current pitch value.  Pitch must maintain the
    same ratio as well as remain between 50\% and 200\%.

  The value of the rate, pitch and speed
  is not persistent, i.e. after the \dap\ is turned on it will
  always be set to 100\%.  However, the rate, pitch and speed
  information will be stored in any bookmarks you may create
  (see \reference{ref:Bookmarkconfigactual}) and will be restored upon
  playing back those bookmarks.

  \begin{btnmap}
    \ActionPsToggleMode
    \opt{HAVEREMOTEKEYMAP}{& \ActionRCPsToggleMode}
    & Toggle pitch changing mode (cycle through all available modes).\\
    %
    \ActionPsIncSmall{} / \ActionPsDecSmall
    \opt{HAVEREMOTEKEYMAP}{& \ActionRCPsIncSmall{} / \ActionRCPsDecSmall}
    & Increase~/ Decrease pitch by 0.1\% (in procentual mode) or 0.1
      semitone (in semitone mode).\\
    %
    \nopt{PBELL_VIBE500_PAD}{ % there is no long scroll up or down because of slide
    \ActionPsIncBig{} / \ActionPsDecBig
    \opt{HAVEREMOTEKEYMAP}{& \ActionRCPsIncBig{} / \ActionRCPsDecBig}
    & Increase~/ Decrease pitch by 1\% (in procentual mode) or a semitone
      (in semitone mode).\\
    }
    %
    \ActionPsNudgeLeft{} / \ActionPsNudgeRight
    \opt{HAVEREMOTEKEYMAP}{& \ActionRCPsNudgeLeft{} / \ActionRCPsNudgeRight}
    & Temporarily change pitch by 2\% (beatmatch), or modify speed (in timestretch mode).\\
    %
    \ActionPsReset
    \opt{HAVEREMOTEKEYMAP}{& \ActionRCPsReset}
    & Reset pitch and speed to 100\%. \\
    %
    \ActionPsExit
    \opt{HAVEREMOTEKEYMAP}{& \ActionRCPsExit}
    & Leave the \setting{Pitch Screen}. \\
    %
  \end{btnmap}

}


%Include playlist section
% $Id$ %
\chapter{Advanced Topics}

\section{\label{ref:CustomisingUI}Customising the User Interface}

\subsection{\label{ref:CustomisingTheMainMenu}Customising The Main Menu}

It is possible to customise the main menu, i.e. to reorder or to hide some
of its items (only the main menu can be customised, submenus can not).
To accomplish this, load a \fname{.cfg} file (as described in
\reference{ref:manage_settings}) containing the following line:
\config{root~menu~order:items}, where ``items'' is a comma separated list
(no spaces around the commas!) of the following
words: \config{bookmarks}, \config{files}, \opt{tagcache}{\config{database}, }%
\config{wps}, \config{settings}, \opt{recording}{\config{recording}, }%
\opt{radio}{\config{radio}, }\config{playlists}, \config{plugins},
\config{system\_menu}, \config{shortcuts}.
Each of the words, if it occurs in the list, activates the appropriate item
in the main menu. The order of the items is given by the order of the words
in the list. The items whose words do not occur in the list will be hidden,
with one exception: the menu item \setting{Settings} will be shown even if
its word is not in the list (it is added as the last item then).

The following configuration example will change the main menu so that it will
contain only the items for the file browser, for resuming the playback, and
for changing the settings (the latter will be added automatically).
\begin{example}
    \config{root menu order:files,wps}
\end{example}


To reset the menu items to the default, use \config{root~menu~order:-} (i.e.
use a hyphen instead of ``items'').

This configuration entry can only be created and edited with a text editor or
the Main Menu Config Plugin (see \reference{ref:main_menu_config}).
It is not possible to change this setting via the settings menu.

\subsection{\label{ref:OpenPlugins}Open Plugin Menu Items}

Rockbox allows you to choose a plugin to run for select menu options.
Simply choose the option in the setting menu and choose the plugin
you would like to run.

\subsection{\label{ref:GettingExtras}Getting Extras}

Rockbox supports custom fonts. A collection of fonts is available for download
in the font package at \url{https://www.rockbox.org/daily.shtml}.

\subsection{\label{ref:Loadingfonts}Loading Fonts}\index{Fonts}
Rockbox can load fonts dynamically. Simply copy the \fname{.fnt} file to the
\dap{} and ``play'' it in the \setting{File Browser}. If you want a font to
be loaded automatically every time you start up, it must be located in the
\fname{/.rockbox/fonts} directory and the filename must be at most 24 characters
long. You can browse the fonts in \fname{/.rockbox/fonts} under
\setting{Settings $\rightarrow$ Theme Settings $\rightarrow$ Font}
in the \setting{Main Menu}.\\

\note{Advanced Users Only: Any BDF font should
  be usable with Rockbox. To convert from \fname{.bdf} to \fname{.fnt}, use
  the \fname{convbdf} tool. This tool can be found in the \fname{tools}
  directory of the Rockbox source code. See \wikilink{CreateFonts\#ConvBdf}
  for more details. Or just run \fname{convbdf} without any parameters to
  see the possible options.}

\subsection{\label{ref:Loadinglanguages}Loading Languages}
\index{Language files}%
Rockbox can load language files at runtime. Simply copy the \fname{.lng} file
\emph{(do not use the .lang file)} to the \dap\ and ``play'' it in the
Rockbox directory browser or select \setting{Settings $\rightarrow$
General Settings $\rightarrow$ Language }from the \setting{Main Menu}.\\

\note{If you want a language to be loaded automatically every time you start
up, it must be located in the \fname{/.rockbox/langs} directory and the filename
must be a maximum of 24 characters long.\\}

If your language is not yet supported and you want to write your own language
file find the instructions on the Rockbox website:
\wikilink{LangFiles}

\opt{lcd_color}{
  \subsection{\label{ref:ChangingFiletypeColours}Changing Filetype Colours}
  Rockbox has the capability to modify the \setting{File Browser} to show
  files of different types in different colours, depending on the file extension.

  \subsubsection{Set-up}
  There are two steps to changing the filetype colours -- creating
  a file with the extension \fname{.colours} and then activating it using
  a config file.  The \fname{.colours} files \emph{must} be stored in
  the \fname{/.rockbox/themes/} directory.
  The \fname{.colours} file is just a text file, and can be edited with
  your text editor of choice.

  \subsubsection{Creating the .colours file}
  The \fname{.colours} file consists of the file extension
  (or \fname{folder}) followed by a colon and then the colour desired
  as an RGB value in hexadecimal, as in the following example:\\*
  \\
  \config{folder:808080}\\
  \config{mp3:00FF00}\\
  \config{ogg:00FF00}\\
  \config{txt:FF0000}\\
  \config{???:FFFFFF}\\*

  The permissible extensions are as follows:\\*
  \\
  \config{folder, m3u, m3u8, cfg, wps, lng, rock, bmark, cue, colours, mpa,
  \firmwareextension{}, %
  mp1, mp2, mp3, ogg, oga, wma, wmv, asf, wav, flac, ac3, a52, mpc, wv,
  m4a, m4b, mp4, mod, shn, aif, aiff, spx, sid, adx, nsf, nsfe, spc, ape,
  mac, sap, mpg, mpeg%
  \opt{HAVE_REMOTE_LCD}{, rwps}%
  \opt{lcd_non-mono}{, bmp}%
  \opt{radio}{, fmr}%
  , fnt, kbd}\\*
  %It'd be ideal to get these from filetypes.c

  All file extensions that are not either specifically listed in the
  \fname{.colours} files or are not in the list above will be
  set to the colour given by \config{???}. Extensions that
  are in the above list but not in the \fname{.colours}
  file will be set to the foreground colour as normal.

  \subsubsection{Activating}
  To activate the filetype colours, the \fname{.colours} file needs to be
  invoked from a \fname{.cfg} configuration file. The easiest way to do
  this is to create a new text file containing the following single
  line:\\*
  \\
  \config{filetype colours: /.rockbox/themes/filename.colours}\\*

  where filename is replaced by the filename you used when creating the
  \fname{.colours} file. Save this file as e.g. \fname{colours.cfg} in the
  \fname{/.rockbox/themes} directory and then activate the config file
  from the menu as normal
  (\setting{Settings} $\rightarrow$ \setting{Theme Settings}%
  $\rightarrow$ \setting{Browse Theme Files}).

  \subsubsection{Editing}
  The built-in \setting{Text Editor} (see \reference{sec:text_editor})
  automatically understands the
  \fname{.colours} file format, but an external text editor can
  also be used. To edit the \fname{.colours} file using Rockbox,
  ``play'' it in the \setting{File Browser}. The file will open in
  the \setting{Text Editor}. Upon selecting a line, the following choices
  will appear:\\*
  \\
  \config{Extension}\\
  \config{Colour}\\*

  If \config{Extension} is selected, the \setting{virtual keyboard}
  (see \reference{sec:virtual_keyboard}) appears,
  allowing the file extension to be modified. If \config{Colour}
  is selected, the colour selector screen appears. Choose the desired
  colour, then save the \fname{.colours} file using the standard
  \setting{Text Editor} controls.
}

\opt{lcd_non-mono}{%
  \subsection{\label{ref:LoadingBackdrops}Loading Backdrops}
  Rockbox supports showing an image as a backdrop in the \setting{File Browser}
  and the menus. The backdrop image must be a \fname{.bmp} file of the exact
  same dimensions as the display in your \dap{} (\dapdisplaysize{} with the last
  number giving the colour depth in bits). To use an image as a backdrop browse
  to it in the \setting{File Browser} and open the \setting{Context Menu}
  (see \reference{ref:Contextmenu}) on it and select the option
  \setting{Set As Backdrop}. If you want rockbox to remember your
  backdrop the next time you start your \dap{} the backdrop must be placed in
  the \fname{/.rockbox/backdrops} directory.
}%

\subsection{UI Viewport}
By default, the UI is drawn on the whole screen. This can be changed so that
the UI is confined to a specific area of the screen, by use of a UI
viewport. This is done by adding the following line to the
\fname{.cfg} file for a theme:\\*

\nopt{lcd_non-mono}{\config{ui viewport: X,Y,[width],[height],[font]}}
\nopt{lcd_color}{\opt{lcd_non-mono}{
    \config{ui viewport: X,Y,[width],[height],[font],[fgshade],[bgshade]}}}
\opt{lcd_color}{
    \config{ui viewport: X,Y,[width],[height],[font],[fgcolour],[bgcolour]}}
\\*

\opt{HAVE_REMOTE_LCD}{
  The dimensions of the menu that is displayed on the remote control of your
  \dap\ can be set in the same way.  The line to be added to the theme
  \fname{.cfg} is the following:\\*

  \nopt{lcd_non-mono}{\config{remote ui viewport: X,Y,[width],[height],[font]}}
  \nopt{lcd_color}{\opt{lcd_non-mono}{
    \config{remote ui viewport: X,Y,[width],[height],[font],[fgshade],[bgshade]}}}
  \opt{lcd_color}{
    \config{remote ui viewport: X,Y,[width],[height],[font],[fgcolour],[bgcolour]}}
\\*

  Only the first two parameters \emph{have} to be specified, the others can
  be omitted using `$-$' as a placeholder. The syntax is very similar to WPS
  viewports (see \reference{ref:Viewports}).  Briefly:

  \nopt{lcd_non-mono}{  \begin{itemize}
    \item `font' is a number: 0 is the built-in system font, 1 is the
    user-selected font.
  \end{itemize}

\begin{example}
    \config{ui viewport: 15,20,100,150,-}
\end{example}
This displays the menu starting at 15px from the left of the screen and 20px
from the top of the screen.  It is 100px wide and 150px high. The font is
defined in the theme \fname{.cfg} file or in the \setting{Theme Settings} menu.

}
  \nopt{lcd_color}{\opt{lcd_non-mono}{  \begin{itemize}
    \item `fgshade' and `bgshade' are numbers in the range 0 (= black) to 3
    (= white).
    \item `font' is a number: 0 is the built-in system font, 1 is the
    user-selected font.
  \end{itemize}

\begin{example}
    \config{ui viewport: 15,20,100,150,-,-,-}
\end{example}
This displays the menu starting at 15px from the left of the screen and 20px
from the top of the screen.  It is 100px wide and 150px high.
The font and the foreground/background shades are defined in the theme
\fname{.cfg} file or in the \setting{Theme Settings} menu.

}}
  \opt{lcd_color}{  \begin{itemize}
    \item `fgcolour' and `bgcolour' are 6-digit RGB888 colours, e.g. FF00FF.
    \item `font' is a number: 0 is the built-in system font, 1 is the
    user-selected font.
  \end{itemize}

\begin{example}
    \config{ui viewport: 15,20,100,150,-,-,-}
\end{example}
This displays the menu starting at 15px from the left of the screen and 20px
from the top of the screen.  It is 100px wide and 150px high.
The font and the foreground/background colours are defined in the theme
\fname{.cfg} file or in the \setting{Theme Settings} menu.

}

\section{\label{ref:ConfiguringtheWPS}Configuring the Theme}

\subsection{Themeing -- General Info}

  There are various different aspects of the Rockbox interface
  that can be themed -- the WPS or \setting{While Playing Screen}, the FMS or
  \setting{FM Screen} (if the \dap{} has a tuner), and the SBS or
  \setting{Base Skin}. The WPS is the name used to
  describe the information displayed on the \daps{} screen whilst an audio
  track is being played, the FMS is the screen shown while listening to the
  radio, and the SBS lets you specify a base skin that is shown in the
  menus and browsers, as well as the WPS and FMS. The SBS also allows you to
  control certain aspects of the appearance of the menus/browsers.
  There are a number of themes included in Rockbox, and
  you can load one of these at any time by selecting it in
  \setting{Settings $\rightarrow$ Theme Settings $\rightarrow$ Browse Theme Files}.
  It is also possible to set individual items of a theme from within the
  \setting{Settings $\rightarrow$ Theme Settings} menu.


\subsection{\label{ref:CreateYourOwnWPS}Themes -- Create Your Own}
The theme files are simple text files, and can be created (or edited) in your
favourite text editor. To make sure non-English characters
display correctly in your theme you must save the theme files with UTF-8
character encoding. This can be done in most editors, for example Notepad in
Windows 2000 or XP (but not in 9x/ME) can do this.

\begin{description}
\item [Files Locations: ] Each different ``themeable'' aspect requires its own file --
  WPS files have the extension \fname{.wps}, FM screen files have the extension
  \fname{.fms}, and SBS files have the extension \fname{.sbs}. The main theme
  file has the extension \fname{.cfg}. All files should have the same name.

  The theme \fname{.cfg} file should be placed in the \fname{/.rockbox/themes}
  directory, while the \fname{.wps}, \fname{.fms} and \fname{.sbs} files should
  be placed in the \fname{/.rockbox/wps} directory. Any images used by the
  theme should be placed in a subdirectory of \fname{/.rockbox/wps} with the
  same name as the theme, e.g. if the theme files are named
  \fname{mytheme.wps, mytheme.sbs} etc., then the images should be placed in
  \fname{/.rockbox/wps/mytheme}.
\end{description}

All full list of the available tags are given in appendix
\reference{ref:wps_tags}; some of the more powerful concepts in theme design
are discussed below.

\begin{itemize}
\item All characters not preceded by \% are displayed as typed.
\item Lines beginning with \# are comments and will be ignored.
\end{itemize}

\note{Keep in mind that your \daps{} resolution is \dapdisplaysize{} (with
  the last number giving the colour depth in bits) when
  designing your own WPS, or if you use a WPS designed for another target.
  \opt{HAVE_REMOTE_LCD}{The resolution of the remote is
      \opt{iriverh100,iriverh300}{128$\times$64$\times$1}%
      \opt{iaudiox5,iaudiom5,iaudiom3}{128$\times$96$\times$2}
      pixels.
  }
}

\subsubsection{\label{ref:Viewports}Viewports}

By default, a viewport filling the whole screen contains all the elements
defined in each theme file. The
\opt{lcd_non-mono}{elements in this viewport are displayed
  with the same background/\linebreak{}foreground
  \opt{lcd_color}{colours}\nopt{lcd_color}{shades} and the}
text is rendered in the
same font as in the main menu. To change this behaviour a custom viewport can
be defined. A viewport is a rectangular window on the screen%
\opt{lcd_non-mono}{ with its own foreground/background
\opt{lcd_color}{colours}\nopt{lcd_color}{shades}}.
This window also has variable dimensions. To
define a viewport a line starting \config{{\%V(\dots)}} has to be
present in the theme file. The full syntax will be explained later in
this section. All elements placed before the
line defining a viewport are displayed in the default viewport. Elements
defined after a viewport declaration are drawn within that viewport.
Loading images (see Appendix \reference{ref:wps_images})
should be done within the default viewport.
A viewport ends either with the end of the file, or with the next viewport
declaration line. Viewports sharing the same
coordinates and dimensions cannot be displayed at the same time. Viewports
cannot be layered \emph{transparently} over one another. Subsequent viewports
will be drawn over any other viewports already drawn onto that
area of the screen.


\nopt{lcd_non-mono}{\subsubsection{Viewport Declaration Syntax}

\config{\%V(x,y,[width],[height],[font]}%

    \begin{itemize}
      \item `font' is a number: 0 is the built-in system font, 1 is the
      current menu font, and 2-9 are additional skin loaded fonts (see 
      \reference{ref:multifont}).
      \item Only the coordinates \emph{have} to be specified. Leaving the other
      definitions blank will set them to their default values.
    \end{itemize}

\note{The correct number of commas with hyphens in
      blank fields are still needed.}
  
\begin{example}
    %V(12,20,-,-,1)
    %sThis viewport is displayed permanently. It starts 12px from the left and
    %s20px from the top of the screen, and fills the rest of the screen from
    %sthat point. The lines will scroll if this text does not fit in the viewport.
    %sThe user font is used.
\end{example}
\begin{rbtabular}{.75\textwidth}{XX}{\textbf{Viewport definition} & \textbf{Default value}}{}{}
  width/height & remaining part of screen \\
  font & user defined \\
\end{rbtabular}

}
\nopt{lcd_color}{\opt{lcd_non-mono}{\subsubsection{Viewport Declaration Syntax}

\config{\%V(x,y,[width],[height],[font]) \%Vf([fgshade]) \%Vb([bgshade])}%

    \begin{itemize}
      \item \%Vf and \%Vb set the foreground and background shade of grey
      respectively.
      \item `fgshade' and `bgshade' are numbers in the range 0 (= black) to 3
      (= white).
      \item `font' is a number: 0 is the built-in system font, 1 is the
      current menu font, and 2-9 are additional skin loaded fonts (see 
      \reference{ref:multifont}).
      \item Only the coordinates \emph{have} to be specified. Leaving the other
      definitions blank will set them to their default values.
    \end{itemize}

\note{The correct number of commas with hyphens in
      blank fields are still needed.}

\begin{example}
    %V(12,20,-,-,1) %Vf(0) %Vb(3)
    %sThis viewport is displayed permanently. It starts 12px from the left and
    %s20px from the top of the screen, and fills the rest of the screen from
    %sthat point. The lines will scroll if this text does not fit in the viewport.
    %sThe user font is used, the foreground colour is set to black and the
    %sbackground is set to white.
\end{example}
\begin{rbtabular}{.75\textwidth}{XX}{\textbf{Viewport definition} & \textbf{Default value}}{}{}
  width/height & remaining part of screen \\
  font & user defined \\
  shade & black foreground on white background \\
\end{rbtabular}

}}
\opt{lcd_color}{\subsubsection{Viewport Declaration Syntax}

\config{\%V(x,y,[width],[height],[font]) \%Vf([fgcolour]) \%Vb([bgcolour]) %
    \%Vg(start, end [,text])}%

    \begin{itemize}
      \item \%Vf and \%Vb set the foreground and background colours respectively.
      \item `fgcolour' and `bgcolour' are 6-digit RGB888 colours, e.g. FF00FF.
      \item \%Vg defines a gradient fill that can then be used with the \%Vs tag.
      `start' and `end' set the initial and final colours, and the optional `text'
      sets the text colour. Colours are 6-digit RGB888, e.g. FF00FF.
      \item `font' is a number: 0 is the built-in system font, 1 is the
      current menu font, and 2-9 are additional skin loaded fonts (see 
      \reference{ref:multifont}).
      \item Only the coordinates \emph{have} to be specified. Leaving the other
      definitions blank will set them to their default values.
    \end{itemize}

\note{The correct number of commas with hyphens in
      blank fields are still needed.}

\begin{example}
    %V(12,20,-,-,1) %Vf(000000) %Vb(FFFFFF) %Vg(FFC0CB, FF0000, FFFF00)
    %sThis viewport is displayed permanently. It starts 12px from the left and
    %s20px from the top of the screen, and fills the rest of the screen from
    %sthat point. The lines will scroll if this text does not fit in the viewport.
    %sThe user font is used, and the foreground and background are set to black
    %sand white respectively. The line gradient is set to pink to red with yellow
    %text.
\end{example}
\begin{rbtabular}{.75\textwidth}{XX}{\textbf{Viewport definition} & \textbf{Default value}}{}{}
  width/height & remaining part of screen \\
  font & user defined \\
  foreground/background colours & defined by theme \\
\end{rbtabular}

}

\opt{lcd_non-mono}{
\subsubsection{Viewport Line Text Styles}
  \begin{tagmap}
    \config{\%Vs(mode[,param])}
    & Set the viewport text style to `mode' from this point forward\\
  \end{tagmap}

Mode can be the following:

\begin{rbtabular}{.75\textwidth}{lX}{\textbf{Mode} & \textbf{Description}}{}{}
  clear & Restore the default style\\
  invert & Draw lines inverted\\
  color & Draw the text coloured by the value given in `param'. Functionally
    equivalent to using the \%Vf() tag\\
  \opt{lcd_color}{%
    gradient & Draw the next `param' lines using a gradient as
    defined by \%Vg. By default the gradient is drawn over 1 line.
    \%Vs(gradient,2) will use 2 lines to fully change from the start colour to
    the end colour\\}
\end{rbtabular}
}
\subsubsection{Conditional Viewports}

Any viewport can be displayed either permanently or conditionally.
Defining a viewport as \config{{\%V(\dots)}}
will display it permanently.

\begin{itemize}
\item {\config{\%Vl('identifier',\dots)}}
This tag preloads a viewport for later display. `identifier' is a single
lowercase letter (a-z) and the `\dots' parameters use the same logic as
the \config{\%V} tag explained above.
\item {\config{\%Vd('identifier')}} Display the `identifier' viewport.
\end{itemize}

Viewports can share identifiers so that you can display multiple viewports
with one \%Vd line.

\nopt{lcd_non-mono}{\begin{example}
    %?mh<%Vd(a)|%Vd(b)>
    %Vl(a,10,10,50,50,-)
    %sYou could now show a hold icon using the %%xl and %%xd tags.
    %Vl(a,0,70,70,14,1)
    %s%acYour DAP is locked.
    %Vl(b,20,14,50,14,1)
    %t(1)%acWarning:;%t(.1) 
    %Vl(b,20,30,50,50,0)
    %sYou've unlocked your player.
\end{example}
This example checks for hold. Viewport `a' will be displayed if it is on,
otherwise viewport `b' will display a flashing warning.
}
\nopt{lcd_color}{%
  \opt{lcd_non-mono}{\begin{example}
    %?C<%Vd(a)|%Vd(b)>
    %Vl(a,10,10,50,50,-)
    %Cl(0,0,50,50,c,c)
    %Cd
    %Vl(a,0,70,70,14,1)
    %s%acThere you have it: Album art.
    %Vl(b,20,14,50,14,1)
    %t(1)%acWarning:;%t(.1) 
    %Vl(b,20,30,50,50,1)
    %sNo album art found
    %scheck your filenames.
\end{example}
This example checks for album art. Album art will be displayed in viewport `a', if
it is found. Otherwise a flashing warning will be displayed in viewport `b'.
}}
\opt{lcd_color}{\begin{example}
    %?C<%Vd(a)|%Vd(b)>
    %Vl(a,10,10,50,50,-)
    %Cl(0,0,50,50,c,c)
    %Cd
    %Vl(a,0,70,70,14,1)
    %s%acThere you have it: Album art.
    %Vl(b,20,14,50,14,1) %Vf(ff0000) %Vb(ffffff)
    %t(1)%acWarning:;%t(.1) 
    %Vl(b,20,30,50,50,1) %Vf(000000) %Vb(ffffff)
    %sNo album art found
    %scheck your filenames.
\end{example}
This example checks for album art. Album art will be displayed in viewport `a', if
it is found. Otherwise a red flashing warning will be displayed in viewport `b'.
}
\\*

\note{The tag to display conditional viewports must come before the tag to
preload the viewport in the \fname{.wps} file.}

\subsection{Info Viewport (SBS only)}
As mentioned above, it is possible to set a UI viewport via the theme
\fname{.cfg} file. It is also possible to set the UI viewport through the SBS
file, and to conditionally select different UI viewports.

  \begin{itemize}
    \item {\config{\%Vi('label',\dots)}}
    This viewport is used as Custom UI Viewport in the case that the theme
    doesn't have a ui viewport set in the theme \fname{.cfg} file. Having this
    is strongly recommended since it makes you able to use the SBS
    with other themes. If label is set this viewport can be selectivly used as the
    Info Viewport using the \%VI tag. The `\dots' parameters use the same logic as
    the \config{\%V} tag explained above.

    \item {\config{\%VI('label')}} Set the Info Viewport to use the viewport called
    label, as declared with the previous tag.
  \end{itemize}

\subsection{\label{ref:multifont}Additional Fonts}
Additional fonts can be loaded within each screen file to be used in that
screen. In this way not only can you have different fonts between e.g. the menu
and the WPS, but you can use multiple fonts in each of the individual screens.\\

\config{\%Fl('id',filename,glyphs)}

  \begin{itemize}
    \item `id' is the number you want to use in viewport declarations, 0 and 1
       are reserved and so can't be used.
    \item `filename' is the font filename to load. Fonts should be stored in
       \fname{/.rockbox/fonts/}
    \item `glyphs' is an optional specification of how many unique glyphs to
       store in memory. Default is from the system setting
       \setting{Glyphs To Load}.
  \end{itemize}

  An example would be: \config{\%Fl(2,12-Nimbus.fnt,100)}

}

\subsubsection{Conditional Tags}

\begin{description}
\item[If/else: ]
Syntax: \config{\%?xx{\textless}true{\textbar}false{\textgreater}}

If the tag specified by ``\config{xx}'' has a value, the text between the
``\config{{\textless}}'' and the ``\config{{\textbar}}'' is displayed (the true
part), else the text between the ``\config{{\textbar}}'' and the
``\config{{\textgreater}}'' is displayed (the false part).
The else part is optional, so the ``\config{{\textbar}}'' does not have to be
specified if no else part is desired. The conditionals nest, so the text in the
if and else part can contain all \config{\%} commands, including conditionals.

\item[Enumerations: ]
Syntax: \config{\%?xx{\textless}alt1{\textbar}alt2{\textbar}alt3{\textbar}\dots{\textbar}else{\textgreater}}

For tags with multiple values, like Play status, the conditional can hold a
list of alternatives, one for each value the tag can have.
Example enumeration:
\begin{example}
     \%?mp{\textless}Stop{\textbar}Play{\textbar}Pause{\textbar}Ffwd{\textbar}Rew{\textgreater}
\end{example}

The last else part is optional, and will be displayed if the tag has no value.
The WPS parser will always display the last part if the tag has no value, or if
the list of alternatives is too short.
\end{description}

\subsubsection{Next Song Info}
You can display information about the next song -- the song that is
about to play after the one currently playing (unless you change the
plan).

If you use the upper-case versions of the
three tags: \config{F}, \config{I} and \config{D}, they will instead refer to
the next song instead of the current one. Example: \config{\%Ig} is the genre
name used in the next song and \config{\%Ff} is the mp3 frequency.\\

\note{The next song information \emph{will not} be available at all
  times, but will most likely be available at the end of a song. We
  suggest you use the conditional display tag a lot when displaying
  information about the next song!}

\subsubsection{\label{ref:AlternatingSublines}Alternating Sublines}

It is possible to group items on each line into 2 or more groups or
``sublines''. Each subline will be displayed in succession on the line for a
specified time, alternating continuously through each defined subline.

Items on a line are broken into sublines with the semicolon
`\config{;}' character. The display time for
each subline defaults to 2 seconds unless modified by using the
`\config{\%t}' tag to specify an alternate
time (in seconds and optional tenths of a second) for the subline to be
displayed.

Subline related special characters and tags:
\begin{description}
\item[;] Split items on a line into separate sublines
\item[\%t] Set the subline display time. The
`\config{\%t}' is followed by either integer seconds (\config{\%t5}), or seconds
and tenths of a second within () e.g. (\config{\%t(3.5)}).
\end{description}

Each alternating subline can still be optionally scrolled while it is
being displayed, and scrollable formats can be displayed on the same
line with non{}-scrollable formats (such as track elapsed time) as long
as they are separated into different sublines.
Example subline definition:
\begin{example}
     %s%t(4)%ia;%s%it;%t(3)%pc %pr : Display id3 artist for 4 seconds,
                                 Display id3 title for 2 seconds,
                                 Display current and remaining track time
                                 for 3 seconds,
                                 repeat...
\end{example}

Conditionals can be used with sublines to display a different set and/or number
of sublines on the line depending on the evaluation of the conditional.
Example subline with conditionals:
\begin{example}
    %?it{\textless}%t(8)%s%it{\textbar}%s%fn{\textgreater};%?ia{\textless}%t(3)%s%ia{\textbar}%t(0){\textgreater}\\
\end{example}

The format above will do two different things depending if ID3 tags are
present. If the ID3 artist and title are present:
\begin{itemize}
\item Display id3 title for 8 seconds,
\item Display id3 artist for 3 seconds,
\item repeat\dots
\end{itemize}
If the ID3 artist and title are not present:
\begin{itemize}
\item Display the filename continuously.
\end{itemize}
Note that by using a subline display time of 0 in one branch of a conditional,
a subline can be skipped (not displayed) when that condition is met.

\subsubsection{Using Images}
You can have as many as 52 images in your WPS. There are various ways of
displaying images:
\begin{enumerate}
  \item Load and always show the image, using the \config{\%x} tag
  \item Preload the image with \config{\%xl} and show it with \config{\%xd}.
    This way you can have your images displayed conditionally.
  \item Load an image and show as backdrop using the \config{\%X} tag. The
    image must be of the same exact dimensions as your display.
\end{enumerate}

\optv{swcodec}{% This doesn't depend on swcodec but we don't have a \noptv
               % command.
  Example on background image use:
  \begin{example}
    %X(background.bmp)
  \end{example}
  The image with filename \fname{background.bmp} is loaded and used in the WPS.
}%

Example on bitmap preloading and use:
\begin{example}
    %x(a,static_icon.bmp,50,50)
    %xl(b,rep\_off.bmp,16,64)
    %xl(c,rep\_all.bmp,16,64)
    %xl(d,rep\_one.bmp,16,64)
    %xl(e,rep\_shuffle.bmp,16,64)
    %?mm<%xd(b)|%xd(c)|%xd(d)|%xd(e)>
\end{example}
Four images at the same x and y position are preloaded in the example. Which
image to display is determined by the \config{\%mm} tag (the repeat mode).

\subsubsection{Example File}
\begin{example}
    %s%?in<%in - >%?it<%it|%fn> %?ia<[%ia%?id<, %id>]>
    %pb%pc/%pt
\end{example}
That is, ``tracknum -- title [artist, album]'', where most fields are only
displayed if available. Could also be rendered as ``filename'' or ``tracknum --
title [artist]''.

%\begin{verbatim}
%  %s%?it<%?in<%in. |>%it|%fn>
%  %s%?ia<%ia|%?d2<%d(2)|(root)>>
%  %s%?id<%id|%?d1<%d(1)|(root)>> %?iy<(%iy)|>
%
%  %al%pc/%pt%ar[%pp:%pe]
%  %fbkBit %?fv<avg|> %?iv<(id3v%iv)|(no id3)>
%  %pb
%  %pm
% 
%\end{verbatim}

\section{\label{ref:manage_settings}Managing Rockbox Settings}

\subsection{Introduction to \fname{.cfg} Files}
Rockbox allows users to store and load multiple settings through the use of
configuration files. A configuration file is simply a text file with the
extension \fname{.cfg}.

A configuration file may reside anywhere on the disk. Multiple
configuration files are permitted. So, for example, you could have
a \fname{car.cfg} file for the settings that you use while playing your
jukebox in your car, and a \fname{headphones.cfg} file to store the
settings that you use while listening to your \dap{} through headphones.

See \reference{ref:cfg_specs} below for an explanation of the format
for configuration files. See \reference{ref:manage_settings_menu} for an
explanation of how to create, edit and load configuration files.

\subsection{\label{ref:cfg_specs}Specifications for \fname{.cfg} Files}

The Rockbox configuration file is a plain text file, so once you use the
\setting{Save .cfg file} option to create the file, you can edit the file on
your computer using any text editor program. See
Appendix \reference{ref:config_file_options} for available settings. Configuration
files use the following formatting rules: %

\begin{enumerate}
\item Each setting must be on a separate line.
\item Each line has the format ``setting: value''.
\item Values must be within the ranges specified in this manual for each
  setting.
\item Lines starting with \# are ignored. This lets you write comments into
  your configuration files.
\end{enumerate}

Example of a configuration file:
\begin{example}
    volume: 70
    bass: 11
    treble: 12
    balance: 0
    time format: 12hour
    volume display: numeric
    show files: supported
    wps: /.rockbox/car.wps
    lang: /.rockbox/afrikaans.lng
\end{example}

\note{As you can see from the example, configuration files do not need to
  contain all of the Rockbox options.  You can create configuration files
  that change only certain settings. So, for example, suppose you
  typically use the \dap{} at one volume in the car, and another when using
  headphones. Further, suppose you like to use an inverse LCD when you are
  in the car, and a regular LCD setting when you are using headphones. You
  could create configuration files that control only the volume and LCD
  settings. Create a few different files with different settings, give
  each file a different name (such as \fname{car.cfg},
  \fname{headphones.cfg}, etc.), and you can then use the \setting{Browse .cfg
    files} option to quickly change settings.\\}

  A special case configuration file can be used to force a particular setting
  or settings every time Rockbox starts up (e.g. to set the volume to a safe
  level). Format a new configuration file as above with the required setting(s)
  and save it into the \fname{/.rockbox} directory with the filename
  \fname{fixed.cfg}.

\subsection{\label{ref:manage_settings_menu}The \setting{Manage Settings}
  menu} The \setting{Manage Settings} menu can be found in the \setting{Main
  Menu}. The \setting{Manage Settings} menu allows you to save and load
  \fname{.cfg} files.

\begin{description}

\item [Browse .cfg Files]Opens the \setting{File Browser} in the
  \fname{/.rockbox} directory and displays all \fname{.cfg} (configuration)
  files. Selecting a \fname{.cfg} file will cause Rockbox to load the settings
  contained in that file. Pressing \ActionStdCancel{} will exit back to the
  \setting{Manage Settings} menu. See the \setting{Write .cfg files} option on
  the \setting{Manage Settings} menu for details of how to save and edit a
  configuration file.

\item [Reset Settings]This wipes the saved settings
  in the \dap{} and resets all settings to their default values.

  \opt{IRIVER_H100_PAD,IRIVER_H300_PAD,IAUDIO_X5_PAD,SANSA_E200_PAD,SANSA_C200_PAD%
      ,PBELL_VIBE500_PAD,SAMSUNG_YH92X_PAD,SAMSUNG_YH820_PAD}{
      \note{You can also reset all settings to their default
      values by turning off the \dap, turning it back on, and holding the
      \ButtonRec{} button immediately after the \dap{} turns on.}
  }
  \opt{IRIVER_H10_PAD}{\note{You can also reset all settings to
      their default values by turning off the \dap, and turning it back on
      with the \ButtonHold{} button on.}
  }
  \opt{IPOD_4G_PAD}{\note{You can also reset all settings to their default
      values by turning off the \dap, turning it back on, and activating the
      \ButtonHold{} button immediately after the backlight comes on.}
  }
  \opt{GIGABEAT_PAD}{\note{You can also reset all settings to their default
      values by turning off the \dap, turning it back on and pressing the
      \ButtonA{} button immediately after the \dap{} turns on.}
  }

\item [Save .cfg File]This option writes a \fname{.cfg} file to
  your \daps{} disk. The configuration file has the \fname{.cfg}
  extension and is used to store all of the user settings that are described
  throughout this manual.

  Hint: Use the \setting{Save .cfg File} feature (\setting{Main Menu
    $\rightarrow$ Manage Settings}) to save the current settings, then
  use a text editor to customize the settings file. See Appendix
  \reference{ref:config_file_options} for the full reference of available
  options.

\item [Save Sound Settings]This option writes a \fname{.cfg} file to
  your \daps{} disk. The configuration file has the \fname{.cfg}
  extension and is used to store all of the sound related settings.

\item [Save Theme Settings]This option writes a \fname{.cfg} file to
  your \daps{} disk. The configuration file has the \fname{.cfg}
  extension and is used to store all of the theme related settings.

\end{description}

\section{\label{ref:FirmwareLoading}Firmware Loading}

\subsection{\label{ref:using_rolo}Using ROLO (Rockbox Loader)}
Rockbox is able to load and start another firmware file without rebooting.
You just ``play'' a file with the extension %
\opt{iriverh100,iriverh300}{\fname{.iriver}.} %
\opt{ipod}{\fname{.ipod}.} %
\opt{iaudio}{\fname{.iaudio}.} %
\opt{sansa,iriverh10,iriverh10_5gb,mrobe100,vibe500,samsungyh}{\fname{.mi4}.} %
\opt{sansaAMS,fuzeplus}{\fname{.sansa}.} %
\opt{gigabeatf,gigabeats}{\fname{.gigabeat}.} %
This can be used to test new firmware versions without deleting your
current version.

\opt{multi_boot}{
\subsection{\label{ref:using_multiboot}Using Multiboot}
  \newcommand{\redirectext}{<playername>}
  \opt{fuze}{\renewcommand*{\redirectext}{fuze}}
  \opt{fuzev2}{\renewcommand*{\redirectext}{fuze2}}
  \opt{fuzeplus}{\renewcommand*{\redirectext}{fuze+}}
  \opt{clipplus}{\renewcommand*{\redirectext}{clip+}}
  \opt{clipzip}{\renewcommand*{\redirectext}{clipzip}}
  \opt{e200}{\renewcommand*{\redirectext}{e200}}

  Using a specially crafted redirect file combined with special firmware builds
  that allow the boot drive to be passed from the bootloader, some devices can
  run Rockbox from a MicroSD card. You can even direct the bootloader to run the
  firmware from a different folder within the SD card if desired.
  \note{
    Each supported model looks for a specific redirect file
    (file extension is unique to each model).
  }

  Create \fname{/rockbox\_main.\redirectext{}} in the root of the SD card.
  The redirect file should contain a single line denoting the path to the root
  of the desired \fname{.rockbox} directory. A single forward slash ``/'' indicates
  \fname{/.rockbox} in the root of the sd card.
  \note{The redirect file does not work when placed on the internal drive.}

  If instead you wanted to run the Rockbox from SD card \fname{/mybuild/.rockbox}
  then your \fname{/rockbox\_main.\redirectext{}} file should contain:
  ``/mybuild/''
}

\section{Optimising battery runtime}
  Rockbox offers a lot of settings that have high impact on the battery runtime
  of your \dap{}. The largest power savings can be achieved through disabling
  unneeded hardware components -- for some of those there are settings
  available.


  Another area of savings is avoiding or reducing CPU boosting
  through disabling computing intense features (e.g. sound processing) or
  using effective audio codecs.
  The following provides a short overview of the most relevant settings and
  rules of thumb.

\subsection{Display backlight}
  The active backlight consumes a lot of power. Therefore choose a setting that
  disables the backlight after timeout (for setting \setting{Backlight} see
  \reference{ref:Displayoptions}). Avoid having the backlight enabled all the
  time (Activating \setting{selectivebacklight}
  \reference{ref:selectivebacklight} can further reduce power consumption).

\opt{lcd_sleep}{
\subsection{Display power-off}
  Shutting down the display and the display controller saves a reasonable amount
  of power. Choose a setting that will put the display to sleep after timeout
  (for setting \setting{Sleep} see \reference{ref:Displayoptions}). Avoid to
  have the display enabled all the time -- even, if the display is transflective
  and is readable without backlight. Depending on your \dap{} it might be
  significantly more efficient to re-enable the display and its backlight for a
  glimpse a few times per hour than to keep the display enabled.
}

\opt{accessory_supply}{
\subsection{Accessory power supply}
  As default your \dap{}'s accessory power supply is always enabled to ensure
  proper function of connected accessory devices. Disable this power supply, if
  -- or as long as -- you do not use any accessory device with your \dap{} while
  running Rockbox (see \reference{ref:AccessoryPowerSupply}).
}

\opt{lineout_poweroff}{
\subsection{Line Out}
  Rockbox allows to switch off the line-out on your \dap{}. If you do not need
  the line-out, switch it off (see \reference{ref:LineoutOnOff}).
}

\opt{spdif_power}{
\subsection{Optical Output}
  Rockbox allows to switch off the S/PDIF output on your \dap{}. If you do not
  need this output, switch it off (see \reference{ref:SPDIF_OnOff}).
}

\opt{disk_storage}{
\subsection{Anti-Skip Buffer}
  Having a large anti-skip buffer tends to use more power, and may reduce your
  battery life. It is recommended to always use the lowest possible setting
  that allows correct and continuous playback (see \reference{ref:AntiSkipBuf}).
}

\subsection{Replaygain}
  Replaygain is a post processing that equalises the playback volume of audio
  files to the same perceived loudness. This post processing applies a factor
  to each single PCM sample and is therefore consuming additional CPU time. If
  you want to achieve some (minor) savings in runtime, switch this feature off
  (see \reference{ref:ReplayGain}).

\subsection{Peak Meter}
  The peak meter is a feature of the While Playing Screen and will be updated with a
  high framerate. Depending on your \dap{} this might result in a high CPU load. To
  save battery runtime you should switch this feature off (see \reference{ref:peak_meter}).
  \opt{ipodvideo}{
    \note{Especially the \playerman{} \playertype{} suffers from an enabled peak meter.}
  }

\subsection{Audio format and bitrate}
  In general the fastest decoding audio format will be the best in terms of
  battery runtime on your \dap{}. An overview of different codec's performance
  on different \dap{}s can be found at \wikilink{CodecPerformanceComparison}.

\opt{flash_storage}{
  Your target uses flash that consumes a certain amount of power during access.
  The less often the flash needs to be switched on for buffering and the shorter
  the buffering duration is, the lower is the overall power consumption.
  Therefore the bitrate of the audio files does have an impact on the battery
  runtime as well. Lower bitrate audio files will result in longer battery
  runtime.
}
\opt{disk_storage}{
  Your target uses a hard disk which consumes a large amount of power while
  spinning -- up to several hundred mA. The less often the hard disk needs to
  spin up for buffering and the shorter the buffering duration is, the lower is
  the power consumption. Therefore the bitrate of the audio files does have an
  impact on the battery runtime as well. Lower bitrate audio files will result
  in longer battery runtime.
}

  Please do not re-encode any existing audio files from one lossy format to
  another based upon the above mentioned. This will reduce the audio quality.
  If you have the choice, select the best suiting codec when encoding the
  original source material.

\subsection{Sound settings}
  In general all kinds of sound processing will need more CPU time and therefore
  consume more power. The less sound processing you use, the better it is for
  the battery runtime (for options see \reference{ref:configure_rockbox_sound}).

% $Id$ %
\opt{hotkey}{
    \section{\label{ref:Hotkeys}Hotkeys}
    Hotkeys are shortcut keys for use in the \nopt{touchscreen}{\setting{File Browser}
    and }\setting{WPS} screen.  To use one, press 
    \nopt{touchscreen}{\ActionTreeHotkey{} within the \setting{File Browser} or}
    \ActionWpsHotkey{} within the \setting{WPS}
    screen.\nopt{touchscreen}{ The assigned function will launch with reference
    to the current file or directory, if applicable.  Each screen has its own
    assignment.} If there is no assignment for a given screen,
    the hotkey is ignored.
    
    The default assignment for the \nopt{touchscreen}{File Browser hotkey is
    \setting{Off}, while the default for the }WPS hotkey is
    \setting{View Playlist}.
    
    The hotkey assignments are changed in the Hotkey menu (see
    \reference{ref:HotkeySettings}) under \setting{General Settings}.
}

